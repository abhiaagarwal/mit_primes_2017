\subsection*{G7}

For each integer $n \geq 1$ let $\tau_n$ denote the set of nondegenerate triangles whose side lengths are in ${1, \dots , n}$. Moreover, for each triangle $\triangle{ABC}$, let:

\begin{gather} \label{eq:g7}
    D(\triangle{ABC}) = min(|AB - AC|, |BC - BA|, |CA - CB|)
\end{gather}

\subsubsection*{Part A}

For each triangle, to maximize the minimum side differences, each triangle should have a constant difference from each side. It doesn't matter if a triangle has one side with a great difference, since $D$ takes a minimum value. Therefore, we can say the sides of the triangle are:

\begin{gather} \label{eq:triangle_equalities}
    a > b > c \\
    a = b - k = c - 2k \\
    \begin{align}
        D(a, b, c) &= min(|a-b|, |a-c|, |b-c|) \\
        &= min(|b-k-b|, |c-2k-c|, |a+k-(a+2k)|)  \\
        &= min(|-2k|, |-2k|, |-k|) = k
    \end{align}
\end{gather}

Where $k$ is the constant difference between each side. To discover the value of $k$, one must utilise triangle inequalities:

\begin{gather} \label{eq:triangle_inequalities}
    a < b + c \\
    a < a + k + a + 2k \\
    -a < 3k \\
    a > -3k \\
    a > 3k \\
    \text{k is a difference, so the negative is absorbed into the variable} \\
    a = n \\
    n > 3k \\
    k < \frac{n}{3} \\
    k = floor\left(\frac{n}{3}\right) \\
    \text{The value has to be adjusted by an index of $1$} \\
    \text{because multiples of 3 can be degenerate cases} \\
    k = floor\left(\frac{n-1}{3}\right) \\
\end{gather}

Subsituting the $k$ value from \ref{eq:triangle_inequalities} into \ref{eq:triangle_equalities} gives:

\begin{gather}
    max(D(\triangle{\tau_n})) = floor\left(\frac{n-1}{3}\right)
\end{gather}

\subsubsection*{Part B}

For which $n$ is this maximum value achieved for a unique triangle in $\tau_n$ (up to congruence)?