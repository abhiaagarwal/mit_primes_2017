\subsection*{G1}

We have $n$ fair six-sided dice, labeled 1 through 6. Let $p_n$ be the probability that when rolled, the product of all $n$ numbers shown is at most 6.

\subsubsection*{Part A}

Compute the value of $p_2$

\begin{proof}
    $S$ is the set of all sides of the dice. There are 36 unique combinations one can get by rolling 2 die.

    
    \begin{gather*}
        \begin{tabular}{llllll}
    (1,1) & (1,2) & (1,3) & (1,4) & (1,5) & (1,6) \\
    (2,1) & (2,2) & (2,3) & (2,4) & (2,5) & (2,6) \\
    (3,1) & (3,2) & (3,3) & (3,4) & (3,5) & (3,6) \\
    (4,1) & (4,2) & (4,3) & (4,4) & (4,5) & (4,6) \\
    (5,1) & (5,2) & (5,3) & (5,4) & (5,5) & (5,6) \\
    (6,1) & (6,2) & (6,3) & (6,4) & (6,5) & (6,6)
        \end{tabular}
    \end{gather*}

    Their products are: 

    \begin{gather*}
        \begin{tabular}{llllll}
    1 & 2 & 3 & 4 & 5 & 6 \\
    2 & 4 & 6 & 8 & 10 & 12 \\
    3 & 6 & 9 & 12 & 15 & 18 \\
    4 & 8 & 12 & 16 & 20 & 24 \\
    5 & 10 & 15 & 20 & 25 & 30 \\
    6 & 12 & 18 & 24 & 30 & 36
        \end{tabular}
    \end{gather*}

    There are 14 cases where $P \leq 6$, so $p_2 = \frac{14}{36} = 38.9\%$
\end{proof}

\subsubsection*{Part B}

Determine $p_n$ for any integer $n \geq 2$.

\begin{proof}
    I was unable to figure out the solution by hand, so I wrote a python program.

    \newpage
    \lstinputlisting[language=Python, firstline=5, lastline=27]{code/g1.py}

    Running the above program with inputs of $1, 2, 3, 4, 5, 6, 7, 8, 9$ mapped to $6, 14, 25, 39, 56, 76, 99, 125, 154$. Using the OEIS, I found that that the sequence is the Second Pentagonal Numbers offset by ${-1}$. Knowing this, I conjectured that the matches for each n is:

    \begin{align} \label{eq:dice_matches_unshifted}
        m_n &= \frac{3n^2 + n}{2} - 1 \\
        &= \frac{(3n-2)(n+1)}{2}
    \end{align}

    \newpage

    However, $m_2$ gave 6, the first index in my found sequence, so I needed to shift \ref{eq:dice_matches_unshifted} up by one index, so:

    \begin{align} \label{eq:dice_matches}
        m_n = \frac{(3n+1)(n+2)}{2}
    \end{align}

    Our sample space for each $n$ is $6^n$, so to get our probability we take \ref{eq:dice_matches} and divide it by the sample space, giving:

    \begin{gather} \label{eq:dice_probability}
        p_n = \frac{m_n}{6^n} = \frac{(3n+1)(n+2)}{2 * 6^n}
    \end{gather}

    Setting $n=2$ into \ref{eq:dice_probability} gives $\frac{28}{72}$, which is equivalent to the answer found in Part A of this problem.

\end{proof}